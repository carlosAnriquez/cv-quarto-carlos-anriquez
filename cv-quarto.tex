%-----------------------------------------------------------------------------------------------------------------------------------------------%
%	The MIT License (MIT)
%
%	Copyright (c) 2015 Jan Küster
%
%	Permission is hereby granted, free of charge, to any person obtaining a copy
%	of this software and associated documentation files (the "Software"), to deal
%	in the Software without restriction, including without limitation the rights
%	to use, copy, modify, merge, publish, distribute, sublicense, and/or sell
%	copies of the Software, and to permit persons to whom the Software is
%	furnished to do so, subject to the following conditions:
%	
%	THE SOFTWARE IS PROVIDED "AS IS", WITHOUT WARRANTY OF ANY KIND, EXPRESS OR
%	IMPLIED, INCLUDING BUT NOT LIMITED TO THE WARRANTIES OF MERCHANTABILITY,
%	FITNESS FOR A PARTICULAR PURPOSE AND NONINFRINGEMENT. IN NO EVENT SHALL THE
%	AUTHORS OR COPYRIGHT HOLDERS BE LIABLE FOR ANY CLAIM, DAMAGES OR OTHER
%	LIABILITY, WHETHER IN AN ACTION OF CONTRACT, TORT OR OTHERWISE, ARISING FROM,
%	OUT OF OR IN CONNECTION WITH THE SOFTWARE OR THE USE OR OTHER DEALINGS IN
%	THE SOFTWARE.
%	
%
%-----------------------------------------------------------------------------------------------------------------------------------------------%


%============================================================================%
%
%	DOCUMENT DEFINITION
%
%============================================================================%
\documentclass[10pt,A4]{article}	


%----------------------------------------------------------------------------------------
%	ENCODING - No usar inputenc con XeLaTeX
%----------------------------------------------------------------------------------------

%----------------------------------------------------------------------------------------
%	LOGIC
%----------------------------------------------------------------------------------------
\usepackage{xifthen}

%----------------------------------------------------------------------------------------
%	FONT - Configuración para XeLaTeX con Latin Modern
%----------------------------------------------------------------------------------------

% Computer Modern es la fuente por defecto en LaTeX
% No necesitamos fontspec si queremos usar CM estándar

% set font default to sans serif
\renewcommand*\familydefault{\sfdefault} 	

% more font size definitions
\usepackage{moresize}		


%----------------------------------------------------------------------------------------
%	PAGE LAYOUT  DEFINITIONS
%----------------------------------------------------------------------------------------
\usepackage[a4paper]{geometry}		
\geometry{top=1.25cm, bottom=-.1cm, left=1.5cm, right=1.5cm} 	

\usepackage{fancyhdr}				
\pagestyle{fancy}
\usepackage[colorlinks, allcolors=sectcol]{hyperref}
%less space between header and content
\setlength{\headheight}{-5pt}		

\setlength{\parindent}{0mm}

%----------------------------------------------------------------------------------------
%	TABLE /ARRAY DEFINITIONS
%---------------------------------------------------------------------------------------- 
%for layouting tables
\usepackage{multicol}			
\usepackage{multirow}

%extended aligning of tabular cells
\usepackage{array}

\newcolumntype{x}[1]{%
>{\raggedleft\hspace{0pt}}p{#1}}%


%----------------------------------------------------------------------------------------
%	GRAPHICS DEFINITIONS
%---------------------------------------------------------------------------------------- 
\usepackage{graphicx}
\usepackage{wrapfig}
\usepackage{float}
\usepackage{tikz}				
\usetikzlibrary{shapes, backgrounds,mindmap, trees}


%----------------------------------------------------------------------------------------
%	Color DEFINITIONS
%---------------------------------------------------------------------------------------- 

\usepackage{color}

%accent color
\definecolor{sectcol}{HTML}{58748F}

%dark background color
\definecolor{bgcol}{RGB}{110,110,110}

%light background / accent color
\definecolor{softcol}{RGB}{225,225,225}


%============================================================================%
%
%
%	DEFINITIONS
%
%
%============================================================================%

%----------------------------------------------------------------------------------------
% 	HEADER
%----------------------------------------------------------------------------------------

% remove top header line
\renewcommand{\headrulewidth}{0pt} 

%remove botttom header line
\renewcommand{\footrulewidth}{0pt}	  	

%remove pagenum
\renewcommand{\thepage}{}	

%remove section num		
\renewcommand{\thesection}{}			


%----------------------------------------------------------------------------------------
%	custom sections
%----------------------------------------------------------------------------------------

% create a coloured box with arrow and title as cv section headline
% param 1: section title
%
\newcommand{\cvsection}[1]
{
	\begin{center}
		\large\textcolor{sectcol}{\textbf{#1}}
	\end{center}
}

%
\newcommand{\metasection}[2]
{
\normalsize{#2} \hspace*{\fill} \normalsize{#1}\\[1pt]
}

%----------------------------------------------------------------------------------------
%	 CV EVENT
%----------------------------------------------------------------------------------------

% creates a stretched box as cv entry headline followed by some paragraphs about 
% the work you did
% param 1:	event time i.e. 2014 or 2011-2014 etc.
% param 2:	event name (what did you do?)
% param 3:	institution (where did you work / study)
% param 4:	list of paragraphs
%
\newcommand{\cvevent}[4]
{

\begin{tabular*}{1\textwidth}{p{13.6cm}  x{3.9cm}}
	\textbf{#2} - \textcolor{bgcol}{#3} &   \vspace{2.5pt}\textcolor{sectcol}{#1}
\end{tabular*}

\vspace{-5pt}
\textcolor{softcol}{\hrule}
\vspace{8pt}
\foreach \desc in {#4}{
	*  \desc\\[3pt]
}
	
\vspace{3pt}
}

% creates a stretched box as 
\newcommand{\cveventmeta}[2]
{
	\mbox{\mystrut \hspace{87pt}\textit{#1}}\\
	#2
}

%----------------------------------------------------------------------------------------
% CUSTOM STRUT FOR EMPTY BOXES
%----------------------------------------- -----------------------------------------------
\newcommand{\mystrut}{\rule[-.3\baselineskip]{0pt}{\baselineskip}}

\begin{document}

\pagestyle{fancy}

%---------------------------------------------------------------------------------------
%	TITLE HEADLINE
%----------------------------------------------------------------------------------------
\vspace{-8pt}
\begin{center}
	\HUGE Carlos Anríquez\\[6pt]
	\Large Sociólogo
\end{center}

\vspace{6pt}

%---------------------------------------------------------------------------------------
%	META SECTION
%----------------------------------------------------------------------------------------
\metasection{}{\textbf{Contacto}: carlos.anriquez@ug.uchile.cl}
\metasection{}{\textbf{Aptitudes}: Estadística, investigación social
cuantitativa, R, Python, gestión de datos, teoría social, visualización
de datos, inglés avanzado.}

\vspace{-2pt}
\textcolor{softcol}{\hrule}
\vspace{6pt}

\normalsize

\vspace{-6pt}
\cvsection{Perfil}

Sociólogo por la Universidad de Chile, con trayectoria en ciencias
sociales cuantitativas e investigación aplicada. Mi experiencia
principal está en transformar datos complejos en insights
significativos, mediante visualizaciones atractivas y modelos
estadísticos que permitan responder preguntas complejas. Trabajo con R y
Python siguiendo principios de ciencia abierta y flujos reproducibles,
documentando análisis y resultados en Quarto para garantizar
transparencia y replicabilidad. Cuento además con dominio avanzado de
inglés y formaciones complementarias en análisis de datos.

\cvsection{Experiencia}

\cvevent{Sep 2025 - Ene 2026}{Pasantía de investigación}{Programa de las Naciones Unidas Para el Desarrollo}{{Proyecto de investigación: el miedo al crímen en Chile.},{Procesamiento y limpieza Encuesta Nacional de Seguridad (ENUSC).},{Análisis descriptivo y visualizaciones usando ponderadores estadísticos.},{Uso de regresiones multinivel para describir efectos a nivel de comuna, barrio e individual.},{Análisis de bibliografía y escritura académica.}}
\cvevent{Jun 2025 - Actualidad}{Asistente de investigación cuantitativo}{Fondecyt 11241304}{{Proyecto de investigación: Re-conociendo las desigualdades en la academia chilena: Un análisis con perspectiva de género interseccional a las trayectorias de personas con doctorado en STEM},{Procesamiento y limpieza de datos.},{Desarrollo de análisis estadísticos multivariados: ecuaciones estructurales, análisis de clases latentes y regresiones multinivel.}}
\cvevent{Oct 2024 - Actualidad}{Asistente de investigación}{Facultad de Comunicación e Imagen - Universidad de Chile}{{Proyecto de investigación: Monitoreo de desórdenes informativos en las elecciones presidenciales chilenas 2025.},{Recopilar y analizar interacciones entre medios de comunicación y usuarios con métodos mixtos: análisis temático, algoritmos de clasificación y técnicas de reducción de dimensiones.}}
\cvevent{Oct 2023 - Oct 2024}{Tesista }{ Fondecyt: Entre la obediencia y la resistencia: cómo se relaciona la ciudadanía con las policías en Chile?}{{Proyecto de investigación: el rol del miedo y la orientación política en la justificación de la violencia en contexto de protesta social.},{Analizar bibliografía, aplicar estadística descriptiva y multivariada, programación en R Studio y creación de documentos reproducibles con Quarto.}}
\cvevent{Ago 2022 – Oct 2023}{Encuestador}{Dirección de Estudios Sociales - Universidad Católica}{{Aplicar diversos instrumentos cuantitativos de investigación social: encuestas de opinión política, satisfacción de uso y estudios de mercado.},{Capacitación a encuestadores en la aplicación de instrumentos cuantitativos de recolección de información por teléfono.}}
\cvevent{Dic 2020 - Mar 2021}{Práctica Profesional}{División de Educación General - Ministerio de Educación }{{Analizar y gestionar políticas públicas en el Ministerio de Educación, buscando un uso eficiente de los recursos disponibles a partir de la información recopilada.},{Explorar y analizar bases de datos de alta complejidad.},{Base de datos Plan de Mejoramiento Escolar (PME). Cuerpo de texto con los objetivos de mejoramiento escolar de cerca de 3000 establecimientos.},{Base de datos Sistema de Alerta Temprana de Deserción Escolar. Registro nacional del 10\% de estudiantes con mayor probabilidad de deserción.}}
\cvevent{Mar 2019 - Dic 2019}{Ayudante de curso políticas públicas}{Universidad de Chile}{{Organizar material bibliográfico, guiar discusiones grupales y ejercicios de resolución de problemas centrados en el análisis de políticas públicas.}}

\cvsection{Educación}

\cvevent{}{Sociologo, Universidad de Chile}{Santiago, Chile}{{Tesis: el rol del miedo y la orientación política en la justificación de la violencia en contexto de protesta},{Uso de modelos estadísticos para probar modelos teóricos de psicología social}}
\cvevent{}{Python avanzado, Academia desafío LATAM}{Santiago, Chile}{{Uso de python para análisis de datos e inteligencia artificial}}

%--------------------------------------------------------------------------------------------------
%	ARTIFICIAL FOOTER (fancy footer cannot exceed linewidth) 
%--------------------------------------------------------------------------------------------------

\null
\vspace*{\fill}
\hspace{-0.25\linewidth}\colorbox{white}{\makebox[1.5\linewidth][c]{\mystrut  \textnormal{\textcolor{sectcol}{}  *  \textcolor{sectcol}{github.com/carlosAnriquez}}}}

\end{document}
